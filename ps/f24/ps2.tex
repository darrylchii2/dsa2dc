%---------change this every homework
\def\yourid{mst3k}
\def\collabs{list your collaborators here}
\def\sources{list your sources here}
% -----------------------------------------------------
\def\duedate{May 18, 2024 at 11:59p}
\def\pnumber{2}
%-------------------------------------

\documentclass[10pt]{article}
\usepackage{dsa2}
\usepackage{tikz-cd}

\begin{document}
\thispagestyle{empty}
\handout
%%%%%%%%%%%%%%%%%%%%%%%%%%%%%%%%%%%%%%%%%%%%%%%%%%%%%%%%

\begin{problem} Proving Quicksort \end{problem}

For this problem, we'll be reviewing how a \textbf{proof by induction} works.  More specifically, we will be using strong induction.  For a refresher on induction, a few good sources are:
\begin{itemize}    
    \item A summary from an offering of our CS2120 course: \\
    \url {https://www.cs.virginia.edu/~emo7bf/cs2120/s2023/techniques.html#proof-by-induction}

    \item A free online textbook (see Chapter 5): \\ 
    \url {https://people.csail.mit.edu/meyer/mcs.pdf}

    \item Wikipedia has a good description and a few example proofs: \\
    \url{https://en.wikipedia.org/wiki/Mathematical_induction}.  
\end{itemize}

To prove the correctness of an algorithm using induction, we must consider the parts of the proof: the base case, the inductive hypothesis, and the inductive step.

We will use induction to prove that Quicksort is correct.  We will let you assume that the \texttt{partition} operation works correctly.  Recall that \texttt{partition(list,first,last)} returns the location $p$ of an item in the sublist \texttt{list[first:last]}, where all items in positions before $p$ are $< \texttt{list[p]}$, and all items after position $p$ are $> \texttt{list[p]}$.  The item at location $p$ is the pivot-value, and \texttt{partition} puts it into its correct position but does not sort what's before it or after it.  After \texttt{partition} is done, Quicksort is called recursively on the sublists before and after the pivot-value.

\begin{enumerate}
    \item In induction we start with the \textbf{base case}.  If $n=1$ (i.e., there is one item in the list), explain why \texttt{partition} and Quicksort produce the correct answer for a list of size $1$.

    \solution{ 
    % Your solution here 
    }
    
    \item We must next make an assumption: our \textbf{inductive hypothesis}.  Describe briefly this assumption.  \textit{For any list of size less than $n$, ...}
        
    \solution{ 
    % Your solution here 
    }
        
    \item \textbf{Inductive step}.  Now we need to use this \textbf{inductive hypothesis} to make an argument that, for a list of size $n$, Quicksort correctly produces a list that's sorted.  Namely, assuming that \texttt{partition} works correctly and the \textbf{inductive hypothesis} is true, write an argument below that shows, for a list of size $n$, the call to \texttt{partition} and the recursive calls to Quicksort correctly sorts that list.
    
    \solution{
    % Your solution here
    }
        
\end{enumerate}

%%%%%%%%%%%%%%%%%%%%%%%%%%%%%%%%%%%%%%%%%%%%%%%%%%%%%%%%
\begin{problem} Sorry-Oh \end{problem}

You are playing a new video game about a character named "Sorry-Oh". Sorry-Oh is currently at a point in the game where he needs to find a way to get from one corner of a rectangular room to the opposite corner. The rectangular room can be of any size $x$ by $y$, where $x$ and $y$ are positive integers greater than $1$. Unfortunately, some cells contain lava. When Sorry-Oh steps into a cell containing lava, he says "Sorry" and takes a certain amount of damage. The room can be visualized as a rectangular grid of cells whose size is $x$ by $y$. For each move, Sorry-Oh can only step to a cell that is either horizontally or vertically adjacent to his current cell. To get from the starting corner to the opposite corner, Sorry-Oh is only allowed to make a total of ($x + y$ - 2) moves. For example, in the grid shown below (which is 3x3), Sorry-Oh would only be allowed to make 4 (3 + 3 - 2) moves.   

Sorry-Oh starts in the top-left corner cell with a damage value of $0$. Each cell contains a number representing a certain amount of damage specific to that cell. Cells that do not contain lava have a value of $0$. Cells that do contain lava have a positive integer value. When moving from one cell to the next, Sorry-Oh's damage will increase when he arrives in the new cell (assuming the new cell has a value greater than 0).

Your goal is to get Sorry-Oh from the starting cell on the top-left (the starting cell will always be the top-left cell and be designated by $S$) to the ending cell on the bottom-right (the ending cell will always be on the bottom-right), and to have the lowest total damage value after arriving at that ending cell.

Given the following example grid - 

\begin{tabularx}{0.4\textwidth} { 
  | >{\centering\arraybackslash}X 
  | >{\centering\arraybackslash}X 
  | >{\centering\arraybackslash}X | }
 \hline
 S & 2 & 4 \\
 \hline
 2  & 3  & 0  \\
\hline
 2  & 2  & 1  \\
\hline
\end{tabularx}

\vspace{5mm}
\noindent An example path could be computed as follows:
\begin{enumerate}
\item Start cell, damage = 0
\item move Right, d = 2 (0 + 2)
\item move Right, d = 6 (2 + 4))
\item move Down, d = 6 (6 + 0))
\item move Down, d = 7 (6 + 1))
\item Total damage from the moves R-R-D-D is 7.
\end{enumerate}

\noindent Another example path:
\begin{enumerate}
\item Start cell, damage = 0
\item move Down, d = 2 (0 + 2)
\item move Right, d = 5 (2 + 3) 
\item move Down, d = 7 (5 + 2))
\item move Right, d = 8 (7 + 1))
\item Total damage from the moves D-R-D-R is 8.
\end{enumerate}

You are given a grid, but will solve the problem by converting the grid to a graph, then using an algorithm that works on graphs.
\begin{enumerate}
    \item Give a brief description of how you would translate the grid to a weighted, directed graph suitable to solve the problem. 
    \solution{ 
    % Your solution here 
    }

    \item Describe an algorithm that can find a path through the graph resulting in the lowest possible damage. 
    \solution{ 
    % Your solution here 
    }

    \item The runtime of your algorithm must be no worse than $O(|E| log |V|)$. Explain the time complexity of your algorithm.
    \solution{ 
    % Your solution here 
    }

\end{enumerate}

%%%%%%%%%%%%%%%%%%%%%%%%%%%%%%%%%%%%%%%%%%%%%%%%%%%%%%%%
\begin{problem} Bipartite \end{problem}

A graph is bipartite if each vertex can be assigned to one of two sets, such that for every edge in the graph $(v_i,v_j)$, the vertices $v_i$ and $v_j$ do not belong to the same set. 

Write an algorithm that takes a connected undirected graph $G = (V,E)$ and returns a list of edges $E'$ than can be removed from $G$ to make it bipartite, i.e., the graph $G = (V, E-E')$ is bipartite. If $G$ is bipartite to begin with, the list you return will be empty.

\solution{
% Your solution here 
}

\end{document}
